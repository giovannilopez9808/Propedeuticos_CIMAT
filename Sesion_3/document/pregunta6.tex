\section{¿Es la ecuación de transición de estados un DDR igual a la de un robot tipo automovil? Justifique su respuesta.}
Visualmente las dos ecuaciones son semejantes ya que pueden ser reducidas a la siguiente expresión:
\begin{equation*}
    \begin{pmatrix}
        \dot{x} \\
        \dot{y} \\
        \dot{\theta}
    \end{pmatrix} =
    \begin{pmatrix}
        cos(\theta) & 0 \\
        sin(\theta) & 0 \\
        0           & 1
    \end{pmatrix} \begin{pmatrix}
        u_1 \\
        u_2
    \end{pmatrix}
\end{equation*}
pero el robot tipo caro y DDR tienen diferente definición para $u_i$, las cuales son:\\

\begin{minipage}{0.5\linewidth}
    \begin{center}
        Robot tipo carro
    \end{center}
    \begin{align*}
        u_1 & = vcos(\phi) \\
        u_2 & = vsin(\phi)
    \end{align*}
    donde
    \begin{itemize}
        \item $\phi$: Ángulo de giro de las ruedas.
        \item $v$: Es la velocidad.
    \end{itemize}
\end{minipage}
\begin{minipage}{0.5\linewidth}
    \begin{center}
        DDR
    \end{center}
    \begin{align*}
        \begin{pmatrix}
            u_1 \\
            u_1
        \end{pmatrix} = \begin{pmatrix}
            v \\
            \omega
        \end{pmatrix} = \begin{pmatrix}
            \frac{\omega_l+\omega_r}{2} \\
            \frac{\omega_l-\omega_r}{2b}
        \end{pmatrix}
    \end{align*}
    donde
    \begin{itemize}
        \item $\omega_l$: Es la velocidad radial de la rueda.
        \item $\omega_r$: Es la velocidad angular de la rueda.
    \end{itemize}
\end{minipage}
en los dos casos comparten que $\theta$ es el ángulo de giro del carro. Al tener esta diferencia en la definición de las $u_i$ las ecuaciones de transicion de los casos son distintos.