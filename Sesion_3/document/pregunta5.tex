\section{¿Cómo es posible unir el nodo X\textsubscript{near} con X\textsubscript{new} en el caso planificación con restricciones cinemáticas, en el contexto de un RRT?}
La manera de efectuarse el algoritmo RRT con restricciones de movimiento es considerando un conjunto de acciones permitidas por las restricciones cinemáticas y con ello generar vertices cercados q\textsubscript{rand} y entre ellos aplicarles las acciones permitidas q\textsubscript{near} y de entre ellos se escoge el punto más cercano y se agrega al arbol de RRT q\textsubscript{new}.